\LevelOneTitle{理想流体运动基础}

\LevelTwoTitle{雷诺数$Re$}

\begin{equation*}
	Re = \dfrac{\rho V L}{\mu}
\end{equation*}

雷诺数是惯性力和粘性力之比,将流动视作无粘流动(理想流动)的必要条件是$Re \gg 1$。

\LevelTwoTitle{理想流体平衡微分方程(欧拉方程)}

\begin{equation}
	\dfrac{\mathrm{D} \vec{V}}{\mathrm{D} t} = \vec{g} - \dfrac{1}{\rho} \nabla p \label{eq4.2}
\end{equation}

本质是牛顿第二定律的应用,压力和重力的合力产生加速度。特别地,若流体加速度为0,这就是第2章中的式\ref{eq2.1},成为静止流体平衡微分方程。

\LevelTwoTitle{沿流线的伯努利方程}

\begin{equation}
	\dfrac{V^2}{2} + gz + \dfrac{p}{\rho} = \text{const}
\end{equation}

\begin{enumerate}
	\item 物理意义
	\begin{enumerate}
		\item $\dfrac{V^2}{2}$表示单位质量流体的动能;
		\vskip 0.1cm
		\item $gz$表示单位质量流体的重力势能;
		\vskip 0.1cm
		\item $\dfrac{p}{\rho}$表示单位质量流体的压力能;
		\vskip 0.1cm
		\item 整个式子表示单位质量流体的机械能沿流线守恒。
	\end{enumerate}
	\item 几何意义(2021·简答)
	\begin{equation}
		\dfrac{V^2}{2g} + z + \dfrac{p}{\rho g} = \text{const}
	\end{equation}
	\begin{enumerate}
		\item $\dfrac{V^2}{2g}$表示速度水头,是不考虑阻力时流体以速度$V$垂直上射的高度;
		\vskip 0.1cm
		\item $z$表示位置水头,是流体质点相对于基准面的位置高度;
		\vskip 0.1cm
		\item $\dfrac{p}{\rho g}$表示压强水头,是产生压强$p$所需的流体住高度。
		\vskip 0.1cm
		\item 整个式子表示理想流体沿同一条流线的总能头为常数,总能头线是水平直线。
	\end{enumerate}
    \item 适用条件(2021·简答)
    \begin{enumerate}
    	\item 理想均质不可压;
    	\item 定常;
    	\item 质量力有势且只有重力;
    	\item 沿同一条流线;
    	\item 无其他能量输入输出。
    \end{enumerate}
\end{enumerate}

\LevelTwoTitle{伯努利方程的应用}

毕托管、虹吸管、文丘里流量计。

\begin{tip}
	本章可能涉及简答题(考查流线伯努利方程的物理意义、几何意义、适用条件,这些一定要清楚)和计算题(伯努利方程的应用),出计算题的概率小一些,会做作业题即可。
\end{tip}
