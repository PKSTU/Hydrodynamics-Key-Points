\LevelOneTitle{流体静力学}

\LevelTwoTitle{流体静压强}

\LevelThreeTitle{流体静压强的特点}

\begin{enumerate}
	\item 方向垂直于作用面,并指向流体内部;
	\item    静止流体任意点处静压强大小与其作用面方位无关,只是作用点位置的函数,即$p = f(x, y, z)$
\end{enumerate}

\LevelThreeTitle{理想流体压强的特点}

无论运动与否,理想流体压强都只是作用点位置的函数,$p = f(x, y, z)$。

\LevelTwoTitle{静止流体平衡微分方程}

\begin{equation}
	\vec{f} - \dfrac{1}{\rho} \nabla p = 0 \label{eq2.1}
\end{equation}

物理意义:单位质量静止流体中压力与质量力平衡。由该方程也可以推知:等压面与质量力处处垂直。

考虑在重力场中,质量力只有$-\va*{z}$方向的重力,于是有

\begin{equation}
	\dv{p}{z} = -\rho g
\end{equation}

若是均质不可压缩流体,则有

\begin{equation}
	p_1 = p_2 + \rho g h
\end{equation}

这个式子表明:

\begin{enumerate}
	\item 铅垂方向上,压强与淹深呈线性关系;
	\item 等压面为水平面。
\end{enumerate}

\LevelTwoTitle{*帕斯卡原理}

\begin{theorem}[帕斯卡原理]
	充满液体的连通器内,一点的压强变化可瞬时传递到整个连通器内。
\end{theorem}

\begin{tip}
	注意此处可以与第11章的微弱扰动波联系起来。实际上压强变化属于微弱扰动,在流体中以音速传播,若认为液体不可压缩,即$E_v \to \infty$,根据$a = \sqrt{E_v/\rho}$可得其传播速度为无穷大。这就是帕斯卡原理只适用于\textbf{充满液体}的连通器内的原因。
\end{tip}

\LevelTwoTitle{三种压强}

\begin{enumerate}
	\item 绝对压强$p$,以完全真空状态为零压强计量的压强;
	\item 计示压强$p_m$,以当地大气压强为基准计量的压强,$p_m = p - p_a$;
	\item 真空压强$p_v$,绝对压强低于当地大气压强时的计示压强的绝对值,$p_v = p_a - p$。
\end{enumerate}

\LevelTwoTitle{*三种测压计的特点}

\begin{enumerate}
	\item 单管测压计
	\begin{enumerate}
		\item 被测压强不能太大;
		\item 只能测量液体压强;
		\item 被测压强必须高于当地大气压强,即无法测得真空压强。
	\end{enumerate}
    \item U型管测压计
    \begin{enumerate}
    	\item 测量范围比较大;
    	\item 可以测液体、气体压强;
    	\item 可以测真空压强;
    	\item 指示液不能与被测液掺混。
    \end{enumerate}
    \item 倾斜式微压计
    放大了测量距离,提高了测量精度,应注意计算方法。
\end{enumerate}

\begin{tip}
	对于本章,必须要掌握的计算题就是较复杂的组合管测压计的计算,可以参考课后题练习。
\end{tip}