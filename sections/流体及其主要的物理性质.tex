\LevelOneTitle{流体及其主要的物理性质}

\LevelTwoTitle{流体与连续介质模型}

\LevelThreeTitle{什么是流体}

\begin{definition}[流体]
	任意微小剪切力持续作用下发生连续变形的物质。
\end{definition}

\begin{tip}
	流体的这条定义明确地指出了其与固体的区别,简单地说,流体不能受剪。
\end{tip}

\LevelThreeTitle{流体质点}

\begin{definition}
	微小特征体,包含大量分子,具有确定的宏观统计特征。
\end{definition}

\begin{tip}
	教材上关于流体质点的描述:包含有足够多的分子因此具有确定的宏观统计特性,同时与流场的特征长度相比线性尺寸又充分小,可以看作一个几何点的流体团。
\end{tip}

\LevelThreeTitle{连续介质模型}

\begin{definition}[连续介质模型]
	流体由无限多流体质点连绵不断组成,质点之间无间隙。
\end{definition}

\LevelThreeTitle{连续介质模型的适用条件}

分子平均自由程$\ll$流动问题特征尺寸。所以对于稀薄气体,或者是激波层内,连续介质模型不再适用。

\LevelTwoTitle{流体的可压缩性}

\LevelThreeTitle{体积弹性模量}

\begin{equation}
	E_v = \rho \qty(\pdv{p}{\rho})_T = -V \qty(\pdv{p}{V})_T
\end{equation}

这是用来衡量流体可压缩性的一个物理量,流体的体积弹性模量越大,表示流体的体积越难被改变(可以理解为越硬,或者越接近于固体的性质),即流体可压缩性越小。

同样的外界条件下,液体的$E_v$比气体的$E_v$大得多,这表明气体比液体更好压缩。

特别地,如果我们认为气体是理想气体,那么根据理想气体状态方程和过程方程,就可以得到如下结论:

\begin{enumerate}
	\item 等温过程中,气体体积弹性模量$E_v = p$;
	\item 等熵过程中,气体体积弹性模量$E_v = \gamma p$,其中$\gamma = c_p / c_v$. 
\end{enumerate}

\LevelThreeTitle{不可压缩流体}

\begin{definition}[不可压缩流体]
	当流体的体积弹性模量$E_v \to \infty$,我们就称该流体不可压缩。这是一种假想的模型,实际流体都具有压缩性。
\end{definition}

一般地,液体可视作不可压缩,但水击、水下爆炸等现象中应当视作可压缩;气体可视作可压缩,但低速且温差小的气体应当视作不可压缩。

特别地,均质不可压缩流体的数学表达为$\rho = \text{const}$. 

\begin{tip}
	此处注意,流体不可压和均质并不是等价的,具体的解释需要用到物质导数相关的知识,你可以点击\ref{3.3.3}来查看。
\end{tip}

\LevelTwoTitle{流体的粘性}

\LevelThreeTitle{什么是流体的粘性}

\begin{definition}[粘性]
	流体抵抗剪切变形(相对运动)的一种固有的属性。
\end{definition}

\begin{tip}
	此处应当注意“固有属性”的涵义,当流体层间无相对运动时,流体对外不表现粘性,但不能认为此时流体失去了粘性。
\end{tip}

\LevelThreeTitle{流体粘性产生的机理及温度对粘性的影响(2021·简答)}

\begin{enumerate}
	\item 对于液体来讲,其粘性源于分子间内聚力,流体团的剪切变形改变了分子间距离,微观上,分子间引力阻止分子间距离改变,宏观上表现为内摩擦抵抗流体变形。当温度升高时,分子间内聚力变弱,于是粘性减小。
	\item 对于气体来讲,其粘性源于分子热运动,流体层间相对运动时,分子热运动导致层间发生动量交换,宏观上依然表现为内摩擦抵抗流体变形。当温度升高时,分子热运动变得更剧烈,于是粘性增大。
\end{enumerate}

\LevelThreeTitle{牛顿内摩擦定律}

\begin{definition}
	粘性切应力(单位:N/m$^2$)与层间速度梯度(角变形率)成正比,而与速度(角变形量)无关,即
	\begin{equation}
		\tau = \mu \dv{u}{y} = \mu \dv{\gamma}{t}
	\end{equation}
\end{definition}

\LevelThreeTitle{动力粘度和运动粘度}

动力粘度用$\mu$表示,单位是Pa$\cdot$s,可以反映流体真实的粘性大小,动力粘度越大,说明流体粘性越大,数学表达为

\begin{equation}
	\mu = \dfrac{\tau}{\dd{u}/\dd{y}}
\end{equation}

运动粘度用$\nu$表示,单位是m$^2$/s,不能反映流体真实的粘性大小,数学表达为

\begin{equation}
	\nu = \dfrac{\mu}{\rho}
\end{equation}

\LevelThreeTitle{牛顿流体和非牛顿流体(2021·选择)}

\begin{definition}[牛顿流体和非牛顿流体]
	符合牛顿内摩擦定律的流体就是牛顿流体,例如水,酒精等大多数纯液体和低速流动的气体;不符合牛顿内摩擦定律的流体就是非牛顿流体,例如纸浆、面糊和泥石流等。
\end{definition}

\LevelThreeTitle{理想流体}

\begin{definition}[理想流体]
	粘度为零的流体就是理想流体,即$\mu = 0$。
\end{definition}

实际流体都具有粘性。

\LevelTwoTitle{*作用在流体上的两种力}

\begin{enumerate}
	\item 质量力,作用于流体的每个质点上,大小与流体质量成正比,例如重力、惯性力。
	\item 表面力,作用于流体封闭界面上,大小与流体表面积成正比,例如压力、摩擦力。
\end{enumerate}